\documentclass[aspectratio=1610]{beamer}
\input{preamble.tex}
\addbibresource{update.bib}
% Presentation info (title, author, date, etc {{{
\title[Research Update]{CEI Group Meeting:\\Project Update Template}
\author{Randy Linderman}
\institute[Duke]{Electrical and Computer Engineering\\Duke University}
\date{\today}
% }}}

\begin{document}

\begin{frame}[plain,noframenumbering]
  \titlepage
\end{frame}

\begin{frame}[t]
  \frametitle{Updates}
  \begin{itemize}
    \item[\notstarted] Haven't started this yet
    \item[\wontfix] Decided not to do this
    \item[\inprogress] Working on this.
    \item[\done] Finished this!
  \end{itemize}
\end{frame}

\section{How to make a nice research update}
\begin{frame}[t]
  \frametitle{Research update guidelines}
  \framesubtitle{Research update structure}
  \begin{itemize}
    \item Generally its a good idea to split the update into sections to give
      the audience an idea of what to expect.
    \item Presentations are 15 minutes, however, they can be shorter. Don't
      present information that is not pertinent to the group.
    \begin{itemize}
      \item Don't attempt to fill the 15 minutes by discussing irrelavent
        papers. This is a waste of everyone's time.
    \end{itemize}
    \item Citations\footfullcite{linderman2025bnp4ood}$^,$\footfullcite{linderman23}$^,$\footfullcite{zhang2021mixture}
      are supported as well as math commands!
      \begin{align*}
        \cN(x|\mu, \sigma^2) &= \frac{1}{\sqrt{2\pi\sigma^2}}\exp\left(-\frac{1}{2}\frac{(x-\mu)^2}{\sigma^2}\right) \\
        \E[x] &= \mu
      \end{align*}
  \end{itemize}
\end{frame}

\begin{frame}[fragile]
  \frametitle{Research update guidelines}
  \framesubtitle{Literature review}
  \begin{columns}
    \begin{column}{0.5\textwidth}
      \begin{small}
        \hspace{-.75em} When presenting papers, discuss them in the context of your research. Do
        not present the paper's method just because it's interesting.

        \vspace{0.5cm} Instead, think about what insights you can glean from the paper to improve
        or guide your own work. Also, think critically about the paper and
        question its assumptions, methods, and results to determine significance.

        \vspace{0.5cm}
        Beyond citing with bibtex it is a good idea to present the authors'
        institution(s) and PI.
      \end{small}
    \end{column}
    \begin{column}{0.5\textwidth}
      \begin{itemize}
        \begin{scriptsize}
        \item Prefer footnote citations so you don't have to
          jump to the end of the presentation to see the citation. However, you
          can include a references section if its helpful.
        \item Note that citations in 2 column mode cause problems with the
          footnotes. To properly cite in 2 column mode, use
          \verb$\footnotemark$. And outside the \texttt{columns} environment,
          use \verb$\footnotetext[1]{\fullcite{citekey}}$.\footnotemark
        \item Otherwise, \verb$\footfullcite{citekey}$.
        \end{scriptsize}
      \end{itemize}
    \end{column}
  \end{columns}
  \footnotetext[4]{\fullcite{zhang2021mixture}}
\end{frame}

\begin{frame}
  \frametitle{Research update guidelines}
  \framesubtitle{Example figure}
  \begin{columns}
    \begin{column}{0.5\textwidth}
      Figures often look best when presented in the \texttt{columns} environment.
    \end{column}
    \begin{column}{0.5\textwidth}
      \begin{figure}
        \centering
        \includegraphics[width=0.5\textwidth]{images/dukeshield.pdf}
        \caption{Duke Shield}
      \end{figure}
    \end{column}
  \end{columns}
\end{frame}

\begin{frame}[t,allowframebreaks]
  \frametitle{References}
  \printbibliography
\end{frame}
\end{document}
% vim: foldmethod=marker : foldlevel=0:
